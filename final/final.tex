\documentclass[conference]{IEEEtran}
\IEEEoverridecommandlockouts
% The preceding line is only needed to identify funding in the first footnote. If that is unneeded, please comment it out.
\usepackage{cite}
\usepackage{amsmath,amssymb,amsfonts}
\usepackage{algorithmic}
\usepackage{graphicx}
\usepackage{textcomp}
\usepackage{xcolor}
\def\BibTeX{{\rm B\kern-.05em{\sc i\kern-.025em b}\kern-.08em
    T\kern-.1667em\lower.7ex\hbox{E}\kern-.125emX}}
\begin{document}

\title{KUKA LBR iiwa --- Adaptive Assembly Analysis}

\author{\IEEEauthorblockN{Arjan Gupta}
\IEEEauthorblockA{\textit{Robotics Engineering} \\
\textit{Worcester Polytechnic Institute}\\
Worcester, MA, USA \\
agupta11@wpi.edu}
}

\maketitle

\begin{abstract}
This paper presents an analysis of the KUKA LBR iiwa robot's
ability to perform rigid-body assembly tasks. The analysis is based on the first YouTube
video presented in the prompt of the final exam. The video shows the robot's ability
to adapt to the environment and perform manufacturing assembly tasks.
\end{abstract}

\begin{IEEEkeywords}
KUKA, LBR iiwa, assembly, analysis
\end{IEEEkeywords}

\section{Introduction}
The KUKA LBR iiwa robot is a 7-axis robot that is capable of performing 
rigid-body assembly tasks, as shown in the YouTube video~\cite{youtube_2014}. As per the data sheet
of the robot, it is capable of performing assembly tasks with a payload of 7--14 kg,
depending on the model.
Its maximum reach is 800--820 mm depending on the model.

\begin{figure}[h!]
\centering
\includegraphics[scale=0.15]{kuka-setup-desc.png}
\caption{KUKA LBR iiwa in its workspace from the video}
\label{kuka-setup-desc}
\end{figure}

In the video, the robot is first shown in its workspace, showing a HRC-suitable
gripper. A drain valve, a connector for the drain valve, and a connection for
the hoses are shown. A still from the video describing the gripper and workspace is shown in Figure~\ref{kuka-setup-desc}.

After the setup is shown, the robot is shown performing the assembly tasks. The first
part shows the utilization of the joint torque sensors for process recognition. The second
part of the video shows the usage of the joint torque sensors for force-controlled
joining processes. The third part of the video shows the safety features of the robot.

Our objective in this paper is to analyze the robot kinematics and dynamics being
used in each of the three parts of the video. We will also describe how one would
simulate the tasks being performed by the robot in MATLAB, using the Robotics
ToolBox.

\section{Materials and Methods}

\subsection{MATLAB setup for robot}

We first describe the MATLAB setup for the robot. We use the Robotics Toolbox for
building the robot model. The robot model is built using the \texttt{rigidBodyTree} MATLAB type.
We use the KUKA LBR iiwa data sheet to populate the bodies and joints of the
\texttt{rigidBodyTree}. The bodies and joints are created using the \texttt{rigidBody} and
\texttt{rigidBodyJoint} functions in the Robotics Toolbox. At this point, we use
the DH parameters to assign fixed transforms to the joints, using the function
\texttt{setFixedTransform}.
The joints are attached to the bodies
using in the following way: \texttt{body1.Joint = joint1}. The bodies are then
attached to the tree structure by using the \texttt{addBody} function. The tree
can then be displayed using the \texttt{showdetails} function.
\subsection{Analysis of first part}

In this part of the video, the robot is shown performing a task where it correctly
finds the gripping point of the drain valve. The robot then picks up the drain valve,
traces the outline of the drain valve to the attached stopping point, and then inserts
the drain valve into the connection for the drain valve. It then repeats the same process
for a second drain valve.

\begin{figure}[h!]
    \centering
    \includegraphics[scale=0.15]{kuka-drain-valve-insertion.png}
    \caption{Robot aiming to insert the first drain valve --- still from the video}
    \label{kuka-drain-valve-insertion}
\end{figure}

There are two key moments in this part of the video. The first is when the robot
uses its torque sensing to arrive at the `stopping point' in tracing the drain
valve, which helps it recognize the correct gripping point. The second is when
the robot is able to recognize the correct insertion point for the drain valve
using its torque sensing.

\bibliography{refs.bib}
\bibliographystyle{IEEEtran}

\end{document}
