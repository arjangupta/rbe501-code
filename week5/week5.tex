\documentclass[conference]{IEEEtran}
\IEEEoverridecommandlockouts
% The preceding line is only needed to identify funding in the first footnote. If that is unneeded, please comment it out.
\usepackage{cite}
\usepackage{amsmath,amssymb,amsfonts}
\usepackage{algorithmic}
\usepackage{graphicx}
\usepackage{textcomp}
\usepackage{xcolor}
\usepackage{gensymb}
\usepackage{listings}
\usepackage[framed,numbered,autolinebreaks,useliterate]{mcode}
\def\BibTeX{{\rm B\kern-.05em{\sc i\kern-.025em b}\kern-.08em
    T\kern-.1667em\lower.7ex\hbox{E}\kern-.125emX}}
\begin{document}

\title{A Probabilistic Model for Demonstrating High Path Planning Success Rate in Autonomous Capsule Robots for Bronchoscopies}

\author{\IEEEauthorblockN{Arjan Gupta}
\IEEEauthorblockA{\textit{Robotics Engineering} \\
\textit{Worcester Polytechnic Institute}\\
Worcester, MA, USA \\
agupta11@wpi.edu}
}

\maketitle

\begin{abstract}
This manuscript describes a probabilistic view of analyzing the
possible paths that an autonomous robot may take when traversing
through bronchi. We will consider a section of
the lung, as a model that can be expanded upon using our work. Using
the technique of rapidly-exploring random trees (RRT),
we aim to lay down some benchmarks that can be used
by clinical experts to meet the goal of the bronchoscopies
they perform.
\end{abstract}


\begin{IEEEkeywords}
    capsule robot, autonomous bronchoscope, path planning, benchmark parameters
\end{IEEEkeywords}

\section{Introduction}
Bronchoscopies play a vital role in diagnosing a range of pulmonary
issues. For example, tumor detection is an important cause for an
endoscopy performed by a pulmonologist. However, low diagnostic
yields is an existing problem in transbronchial biopsies
\cite{Rivera2013},\cite{WangMemoli2012},\cite{Mehta2018},\cite{Ost2016}.
Among the many solutions to address this
problem, robotic-assisted bronchoscopy is a promising
and ever-growing method\cite{Lu2021}. In the variety of types of robotic
bronchoscopies, a capsule-robot has potential and can be expanded upon in
many ways. For example, the existence of the PillCam COLON capsule has
shown promise in the last decade\cite{Adler2011}. Although again in
the realm of colonoscopy screenings, a capsule endoscopy has shown
to be considerably more cost-effective as compared to traditional
methods\cite{Hassan2008}.\\

While rigid and fiberoptic 
methods remain the dominant methods of performing pulmonary
endoscopies\cite{Lu2021}, they also require highly skilled
operators in order to be safe\cite{Stahl2015}. However,
given the forecasted shortage in the physician workforce
of the United States\cite{Zhang2020}, it is imperative
to look for alternative solutions. Autonomous endoscopies
are one such solution. In fact, autonomous capsule robots
have already been trained using reinforcement learning
for usage in endoscopies\cite{Turan2019}.\\

However, at
the time of the scribing of this manuscript, the existing
literature does not show any preliminary benchmark information
about this type of robotic bronchoscopy. If clinical operators
hope to use autonomous capsule robots for bronchoscopies, there needs
to exist guidelines on tuning the autonomous robot such that a
high success rate for biopsy yield can be achieved.
\textit{
The primary objective of this study is to show a
simple probabilistic method
to achieve over 98\% success rate of an autonomous capsule
robot to find a given destination in the human bronchi.
We will use rapidly-exploring random trees to set up a model that
can establish parameters which can be tuned to achieve a high success
rate.}

\section{Materials and Methods}

\subsection{Set up for the model}
First we use a simple figure that delineates occupied and free zones
of a human pulmonary region. Such a figure is shown in Fig~\ref{given-lung-fig}.
The green region is the the area where the capsule robot is free to move, and
the black regions are the `occupied' zones, signifying the walls of the
organ.

\begin{figure}[h]
    \centering
    \includegraphics[scale=0.11]{939-Oblique.png}
    \caption{A basic model of the human lung}
    \label{given-lung-fig}    
\end{figure}

The capsule robot's dimensions are roughly 10 mm long capsule with 5 mm in diameter,
but for the sake of model simplicity, the dimensions are ignored in the
path planning. We assume that the robot can start anywhere near the trachea
and have a goal position near the end of the narrow bronchi.

\subsection{Occupancy grid and start/goal positions}

Next, we read in the image using MATLAB's `imread' function. Now
that we have a raw byte version of the image, we convert it to
a gray-scale image. By default MATLAB reads this in as the colored
portion marked as black, so our next step is to invert the image,
which we can do with a simple logical invert on the image matrix.



\subsubsection{Restate objective in technical terms}
In technical terms, we must first assign frames 0 through 3 for
each link of the manipulator. We will then form a table of DH
parameters. Using the table, and the general form of the DH matrix,
we will find the $A$ matrices for the manipulator i.e.\ we need to find
$A_1$, $A_2$, and $A_3$. Using these, we need to find $T^0_3$
to give us our final answer. From our textbook~\cite{Spong2006}, the 
DH Coordinate Frame Assumptions are,
\begin{itemize}
    \item \textbf{(DH1)} The axis $x_i$ is perpendicular to the axis $z_{i-1}$.
    \item \textbf{(DH2)} The axis $x_i$ intersects the axis $z_{i-1}$.
\end{itemize}
\vspace{0.1in}

\vspace{0.1in}
\subsubsection{Assign frames}
The first step toward solving our problem is to redraw the robot
manipulator in symbolic
form, and assign frames for links 0 through 3. Since we are following the DH assumptions,
we must follow the frame assignment style shown in Figure 4, which
is from our textbook~\cite{Spong2006}. Our redrawn figure is shown in Figure 5.\\
As shown, frame 0 ($x_0 y_0 z_0$) is assigned at the first joint (revolute). Frame 1
($x_1 y_1 z_1$) is also assigned at the
first joint in order to have DH assumptions satisfied. Frame 2 ($x_2 y_2 z_2$)
 is assigned at third joint (revolute). Frame 3 ($x_3 y_3 z_3$) is assigned at
the end of link 3, in a way that also satisfies DH assumptions.\\

\subsubsection{Create DH table and set variables/constants}
Now that we have assigned the frames, we can use Figure 4 to write
the $\alpha_i$, $a_i$, $\theta_i$, $d_i$ quantities for each link.

\begin{table}[h!]
    \begin{center}
        \resizebox{0.8\columnwidth}{!}{%
        \begin{tabular}{||c|c|c|c|c||}
        \hline
        Link & $\alpha_i$ & $a_i$ & $\theta_i$ & $d_i$ \\
        \hline\hline
        1 & $-90\degree$ & 0 & $\theta_1$ & 0 \\
        2 & $90\degree$ & 0 & 0 & $L_1 + q_2$ \\
        3 & 0 & $L_3$ & $90\degree + \theta_3$ & 0\\
        \hline
        \end{tabular}
        }
    \end{center}
    \caption{Denavit-Hartenberg table for Problem 3--5}
\end{table}

As seen in Table 1, we have chosen $L_1$ and $L_3$ to describe the constant link
lengths of links 1 and 3, respectively. Additionally, $q_2$ describes the variable
link length of link 2 (because of joint 2 being a prismatic joint).
Furthermore, $\theta_1$ and $\theta_3$ describe the
variable angles of revolute joints 1 and 3, respectively.

\subsubsection{Find $A$ matrices}

As given in our textbook~\cite{Spong2006}, the general form of an
$A_i$ matrix is,

\[
    A_i =
    \begin{bmatrix}
        c_{\theta_i} & -s_{\theta_i}c_{\alpha_i} & s_{\theta_i}s_{\alpha_i} & a_i c_{\theta_i}\\
        s_{\theta_i} & c_{\theta_i}c_{\alpha_i} & -c_{\theta_i}s_{\alpha_i} & a_i s_{\theta_i}\\
        0 & s_{\alpha_i} & c_{\alpha_i} & d_i\\
        0 & 0 & 0 & 1
    \end{bmatrix}
\]

Where $c_{\theta_i}$ is $\cos{\theta_i}$, $s_{\theta_i}$ is $\sin{\theta_i}$,
$c_{\alpha_i}$ is $\cos{\alpha_i}$, and $s_{\alpha_i}$ is $\sin{\alpha_i}$.
We can write a MATLAB function for this $A_i$ matrix, as follows.


Using this function in our MATLAB Live Script, we produce the following A matrices.

\[
    A_1 =
    \begin{bmatrix}
        \cos \theta_1 & 0 & -\sin \theta_1 & 0\\
        \sin \theta_1 & 0 & \cos  \theta_1 & 0\\
        0 & -1 & 0 & 0\\
        0 & 0 & 0 & 1
        \end{bmatrix}
\]

\[
    A_2 = 
    \begin{bmatrix}
        1 & 0 & 0 & 0\\
        0 & 0 & -1 & 0\\
        0 & 1 & 0 & L_1 + q_2 \\
        0 & 0 & 0 & 1
        \end{bmatrix}
\]

\[
    A_3 =
    \begin{bmatrix}
        \cos \left(\theta_3 +\frac{\pi }{2}\right) & -\sin \left(\theta_3 +\frac{\pi }{2}\right) & 0 & L_3 \,\cos \left(\theta_3 +\frac{\pi }{2}\right)\\
        \sin \left(\theta_3 +\frac{\pi }{2}\right) & \cos \left(\theta_3 +\frac{\pi }{2}\right) & 0 & L_3 \,\sin \left(\theta_3 +\frac{\pi }{2}\right)\\
        0 & 0 & 1 & 0\\
        0 & 0 & 0 & 1
        \end{bmatrix}
\]

Now we are ready to multiply these matrices to obtain the final result
of our objective.

\section{Results}

\subsection{Result for Problem 3--2}

For Problem 3--2, we obtained our final result by multiplying all three $H$ matrices
that we obtained in the Materials and Methods section. Hence, we obtained the following
matrix,
\[
    H^0_3 =
    \begin{bmatrix}
        \sigma_1  & 0 & \sigma_5  & \lambda_1\\
        0 & 1 & 0 & 0\\
        \sigma_4  & 0 & \sigma_1  & \lambda_2\\
        0 & 0 & 0 & 1
    \end{bmatrix}
\]

where,
\begin{multline*}
    \lambda_1 = L_1 \,\cos \left(\theta_1 \right)+L_3 \,\cos \left(\theta_3 \right)\,\sigma_2 \\-L_3 \,\sin \left(\theta_3 \right)\,\sigma_3 +L_2 \,\cos \left(\theta_1 \right)\,\cos \left(\theta_2 \right)\\-L_2 \,\sin \left(\theta_1 \right)\,\sin \left(\theta_2 \right)
\end{multline*}
\begin{multline*}
    \lambda_2 = -L_1 \,\sin \left(\theta_1 \right)-L_3 \,\cos \left(\theta_3 \right)\,\sigma_3 -L_3 \,\sin \left(\theta_3 \right)\,\sigma_2\\ -L_2 \,\cos \left(\theta_1 \right)\,\sin \left(\theta_2 \right)-L_2 \,\cos \left(\theta_2 \right)\,\sin \left(\theta_1 \right)
\end{multline*}
and,
\begin{align*}
    \sigma_1 &=\cos \left(\theta_3 \right)\,\sigma_2 -\sin \left(\theta_3 \right)\,\sigma_3 \\
    \sigma_2 &=\cos \left(\theta_1 \right)\,\cos \left(\theta_2 \right)-\sin \left(\theta_1 \right)\,\sin \left(\theta_2 \right)\\
    \sigma_3 &=\cos \left(\theta_1 \right)\,\sin \left(\theta_2 \right)+\cos \left(\theta_2 \right)\,\sin \left(\theta_1 \right)\\
    \sigma_4 &= -\cos \left(\theta_3 \right)\,\sigma_3 -\sin \left(\theta_3 \right)\,\sigma_2\\
    \sigma_5 &= \cos \left(\theta_3 \right)\,\sigma_3 +\sin \left(\theta_3 \right)\,\sigma_2
\end{align*}

The significance of this $H^0_3$ matrix is that is provides
a direct transformation matrix between the base frame (frame 0) and end-effector frame,
which is a solution for forward kinematics.

\subsection{Result for Problem 3--5}

For Problem 3--5, we obtained our final result by multiplying all three $A_i$ matrices
that we obtained in the Materials and Methods section. Hence, we obtained the following
matrix,

\[
    T^0_3 =
    \begin{bmatrix}
        \sigma_1  & \sigma_4 & 0 & \lambda_1\\
        \sigma_5 & \sigma_1  & 0 & \lambda_2\\
        0 & 0 & 1 & 0\\
        0 & 0 & 0 & 1
    \end{bmatrix}\\
\]
where,
\begin{align*}
    \lambda_1 &= L_3 \,\cos \left(\theta_1 \right)\,\sigma_3 -\sin \left(\theta_1 \right)\,{\left(L_1 +q_2 \right)}-L_3 \,\sin \left(\theta_1 \right)\,\sigma_2\\
    \lambda_2 &= \cos \left(\theta_1 \right)\,{\left(L_1 +q_2 \right)}+L_3 \,\cos \left(\theta_1 \right)\,\sigma_2 +L_3 \,\sigma_3 \,\sin \left(\theta_1 \right)
\end{align*}
and,
\begin{align*}
    \sigma_1 &=\cos \left(\theta_1 \right)\,\sigma_3 -\sin \left(\theta_1 \right)\,\sigma_2 \\
    \sigma_2 &=\sin \left(\theta_3 +\frac{\pi }{2}\right)\\
    \sigma_3 &=\cos \left(\theta_3 +\frac{\pi }{2}\right)\\
    \sigma_4 &= -\cos \left(\theta_1 \right)\,\sigma_2 -\sigma_3 \,\sin \left(\theta_1 \right)\\
    \sigma_5 &= \cos \left(\theta_1 \right)\,\sigma_2 +\sigma_3 \,\sin \left(\theta_1 \right)
\end{align*}

The significance of this $T^0_3$ matrix is that is provides
a direct transformation matrix between the base frame (frame 0) and end-effector frame, 
which is a solution for forward kinematics.

\section{Discussion}
In the opinion of the author, this homework problem set was insightful. The
first problem proved that we do not need to always use the DH convention
when solving for forward kinematics in robotic manipulators. In fact, when
using tools like MATLAB, manually executing a non-DH method of
computing the forward kinematics is no more complex than using the DH method
itself.\\
The second problem reinforced our learnings from RBE 500. We used the DH
convention heavily in that class, so it was great to revisit that foundation
as we move forward in this class.\\
A topic for further consideration could be, when would one prefer to use
a non-DH method over the DH method? The DH convention can provide a minimal
and efficient way to represent and compute the relationship between the base
frame and the end effector in many cases, because it reduces the number of
variables involved from 6 to 4. However, suppose we want to model the
differential kinematics of a manipulator. The screw-based theory~\cite{Rocha2011}
can provide advantages in such a case. In the referenced paper for screw-based
theory, it was found that, when
various kinematic modelings for common manipulator configurations where
compared, the screw-based theory did not provide any disadvantages in any
case. The one noticeable difference was that it provided superior flexibility
when differential kinematics was compared.
The parameter identification is also a bit simpler in the screw-based theory, as
compared to the DH-convention.\\
It was a great exercise the solve this week's problem set, and the author thanks
the Professor for this.
\bibliography{refs.bib}
\bibliographystyle{IEEEtran}

\end{document}