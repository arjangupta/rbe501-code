\documentclass[conference]{IEEEtran}
\IEEEoverridecommandlockouts
% The preceding line is only needed to identify funding in the first footnote. If that is unneeded, please comment it out.
\usepackage{cite}
\usepackage{amsmath,amssymb,amsfonts}
\usepackage{algorithmic}
\usepackage{graphicx}
\usepackage{textcomp}
\usepackage{xcolor}
\usepackage{gensymb}
\def\BibTeX{{\rm B\kern-.05em{\sc i\kern-.025em b}\kern-.08em
    T\kern-.1667em\lower.7ex\hbox{E}\kern-.125emX}}
\begin{document}

\title{RBE 501 Week 7 Assignment}

\author{\IEEEauthorblockN{1\textsuperscript{st} Arjan Gupta}
\IEEEauthorblockA{\textit{Robotics Engineering} \\
\textit{Worcester Polytechnic Institute}\\
Worcester, MA, USA \\
agupta11@wpi.edu}
}

\maketitle

\begin{abstract}
This is the abstract
\end{abstract}

\begin{IEEEkeywords}
robotics
\end{IEEEkeywords}

\section{Introduction}
We are asked to solve problem x and y.

\section{Materials and Methods}

We know from the textbook~\cite{Spong2006} that,

\[
    \frac{1}{2} \dot{q}^T \left\{ I_1\begin{bmatrix}
        1 & 0\\
        0 & 0
    \end{bmatrix} +
    I_2\begin{bmatrix}
        1 & 1\\
        1 & 1
    \end{bmatrix}\right\}\dot{q}
\]

is the same as,

\[
    \frac{1}{2}I_1\dot{\theta_1}^2 + \frac{1}{2}I_1(\dot{\theta_1} + \dot{\theta_2})^2
\]

\section{Results}
These are the results.

\section{Discussion}

Place author's opinions here.
\bibliography{refs.bib}
\bibliographystyle{IEEEtran}

\end{document}